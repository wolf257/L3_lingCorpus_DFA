
\section{Organisation des dossiers}

\fbox{
\begin{minipage}[t]{.4\textwidth}
\footnotesize{
\begin{center}
AVANT EXÉCUTION DU PROGRAMME
\end{center}

\begin{itemize}
\item \textbf{PROJECT\_ROOT/}
	\begin{itemize}
	\item {src/}
		\begin{itemize}
		\item {modules/}
		\end{itemize}
	\item {corpus\_phrases/}
	\item {from\_outside\_treetagger/}
	\item {rapport/}
	\end{itemize}
\end{itemize}
}
\end{minipage} 
} \fbox{
\begin{minipage}[t]{.4\textwidth}
\footnotesize{ 
\begin{center}
APRÈS EXÉCUTION DU PROGRAMME
\end{center}

\begin{itemize}
\item \textbf{PROJECT\_ROOT/}
	\begin{itemize}
	\item {src/}
		\begin{itemize}
			\item {modules/}
		\end{itemize}
	\item {corpus\_phrases/}
		\begin{itemize}
		\item \textbf{resultats/}
		\end{itemize}
	\item {from\_outside\_treetagger/}
	\item {rapport/}
	\end{itemize}
\end{itemize}
}
\end{minipage}
} 

~\\


%------------------------------------------------------------
%\newpage
\section{Les modules}

La partie programmée est regroupée dans le dossier \emph{src}. \\


 \begin{wrapfigure}{l}{.45\textwidth} % { r, l, i } pour la position,
\fbox{
\begin{minipage}{.40\textwidth}
\footnotesize{
\textbf{code\_source/}
	\begin{itemize}
	\item {main.py}
	\item {settings.py}
	\item {modules/}
		\begin{itemize}
		\item {l1\_big\_process.py} \\

		\item {l2\_tagging.py}
		\item {l2\_transitions.py}
		\item {l2\_others.py} \\

		\item {l3\_pos.py}
		\item {l3\_table.py}
		\end{itemize}
	\end{itemize}
}
\end{minipage}
} 
\end{wrapfigure}

~\\ 

\subsubsection{Rappels}

De même, nous aimerions préciser la logique de notre mouvement (flow). C'est en la suivant que j'ai pu, de manière systématique naviguer à travers le code. \\

En plus des messages souvent clairs de l'interprète de commande, je pars toujours du niveau 0 vers le niveau 2. \\  

Au début de chaque fichier.py du module se trouve la \textbf{liste des fonctions} qu'il contient. \\ \\

\subsubsection{Description des modules} 

\underline{Niveau 0} : Si ce n'est pour contrôler la présentation, le niveau zéro fut très peu sollicité en cas de bug. 
 
	\begin{itemize}
	\item \textbf{main.py} : Contient l'interface. C'est le fichier qui englobe le tout.
	\item \textbf{settings.py} : Pour faciliter la \emph{gestion des liens}, nous avons regroupé les liens vers nos dossiers dans ce fichier. Il ne contient aucune fonction, mais juste des variables. \\
	\end{itemize}

\underline{Niveau 1} : Ce niveau est toujours sollicité, car c'est lui qui coordonne le programme, mais aussi donne les paramètres aux différentes fonctions du niveau 2. C'est le niveau le plus délicat à manipuler.

	\begin{itemize}
	\item \textbf{modules/l1\_big\_process.py} : Ne contient que des \emph{fonctions composées d'autres fonctions} des sous-modules. C'est le seul fichier du dossier modules/ qu'appelle le main. Il sert d'interface entre le main.py et les autres modules. \\
	\end{itemize}

\underline{Niveau 2} : contient toutes les fonctions élémentaires. Quand un problème apparait, il y a de fortes chances qu'un des fichiers de ce niveau doive être vérifié après le niveau 1.

	\begin{itemize}
	\item \textbf{modules/l2\_tagging.py} : contient toutes les fonctions aidant à \emph{l'étiquettage morphosyntaxique} des mots.
	
	\item \textbf{modules/l2\_transitions.py} : contient l'automate à proprement parler.

	\item \textbf{modules/l2\_others.py} : contient diverses fonctions comme celle permettant la creation des dossiers.
			
    \end{itemize}

\underline{Niveau 3} : Ce niveau, le plus bas, est composé de deux fichiers :

	\begin{itemize}
	\item \textbf{modules/l3\_pos.py} : s'occupe de prendre les réponses de treetagger et de nous rendre une catégorie syntaxique selon des règles que nous avons précisées. \\
	\item \textbf{modules/l3\_table.py} : contient les transitions qui sont utilisées par l'automate. \\
	\end{itemize}


%-------------------------------------------------------------
%\newpage
\section{Processus}

\emph{Ce que nous dirons pour un élément (phrases, ... , mot) est valable pour tous ses éléments-frères.} 

\begin{itemize}
%================================================
\item \textbf{- [x] } : NIVEAU GLOBAL
\begin{enumerate}
\item \textbf{[x]} : Création dossier \emph{resultats}.
\item \textbf{[x]} : Récupération liste des phrases des fichiers \emph{phrases\_*} où \{*\} est la langue (en pour english et fr pour français).
\item \textbf{[x]} : Envoi des données et des paramétres.
\end{enumerate}
%================================================
\item \textbf{- [x] } : NIVEAU PHRASES et MOTS
\begin{enumerate}
\item \textbf{[x]} : Découpage de la phrase et récupération des mots sous formes de dictionnaire.
\item \textbf{[x]} : Étiquettage des mots et remplissage du dictionnaire (numéro, pos, lemme).
\item \textbf{[x]} : Utilisation de l'automate pour déterminer si la phrase passe le test ou non.
\item \textbf{[x]} : Écriture des résultats.
\end{enumerate}

\end{itemize}





  