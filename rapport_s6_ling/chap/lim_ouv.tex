 \section{Limites}

Si nous sommes un peu content du programme, nous sommes aussi conscient de ses limites. \\

Parmi les limites de notre automate  : 
\begin{itemize}
\item Il ne peut analyser que des phrases à deux membres maximum (A et B dans le cas d'une structure comparativw. A dans une phrase à structure superlative.). \\

\item Il ne peut analyser les phrases interrogatives pour le moment, surtout en anglais (ex : Is it bigger ? ). \\
 
\item Il n'est pas exhaustif. À l'échelle d'une langue, 200 phrases ou cas ne représentent pas grand chose. \\

\item Au sein même des motifs reconnus, il y a possibilité que certains puissent entrer en conflit avec d'autres. Un corpus plus grand et du temps auraient pu aider. \\

\item Enfin, comme toujours, nous aurions aimé affiner le programme, proposer plus d'options.
\end{itemize}

\section{Ouverture et réflexions diverses}

Toutes les conclusions de notre devoir du Semestre 5 s'appliquent à nouveau ici.

De plus, programmer cet automate m'a permis de me rendre compte de ce que peut être un travail linguistique (le fait de répertorier des cas, de les analyser, de modifier ou ajouter des transitions au fur et à mesure). \\

Ce fut un grand exercice d'observation voire de méditation. On pense tous savoir ce qu'est une chose (ici une comparaison). La formaliser est une autre paire de manche. J'ai passé des heures à juste essayer de comprendre comment mon cerveau analyse une phrase et en déduit si elle est ou non comparative. Ce fut clairement le plus intéressant. \\

Enfin, si nous ne devions garder qu'une phrase de cette année, ce serait \textbf{«Testez votre programme sur un corpus.»}. \\









